\documentclass{paper}[11pt]

\usepackage{amsmath, amssymb, mathtools}
\usepackage{graphicx}
\usepackage{physics}
\usepackage{hyperref, cleveref}
\usepackage{moreverb}
\renewcommand\verbatimtabsize{4\relax}

%%% DEFINING COMMANDS
\newcommand{\bv}[1]{\textbf{#1}}
\newcommand{\I}{\mathbb{I}}
%\newcommand{\Tr}[1]{\text{Tr}}

%%% HEADER STUFF
\title{Some Thoughts for Oscar}
\author{Alec Roberson}

%%% START OF DOCUMENT
\begin{document}
	\maketitle
	
	\section*{Intro}
	
	What does it mean for a two qubit quantum state sampling to be \textit{random}? As you see in the paper I sent \href{https://link.springer.com/article/10.1007/s13538-015-0367-2}{(link here)} there are a number of ways to conceptualize the ``space" that two-qubit quantum states take up, with all of them boiling down to some conception of ``distance" between $n\times n$ matrices. One of the most common definitions of distance (heavily used in the paper)is the Hilbert-Schmidt distance between points, defined as
	\begin{equation}
		\text{HSD}(A, B) = ||A-B||_2
	\end{equation}
	Where the Hilbert-Schmidt norm is
	\begin{equation}
		||A||_2 = \sqrt{\braket{A}{A}}
	\end{equation}
	And the Hilbert-Schmidt inner product of two matrices is defined as
	\begin{equation}
		\braket{A}{B} = \Tr(A^\dagger B)
	\end{equation}
	
	My thoughts going forward are a bit half-baked but I think I am onto something quite interesting, so I'll do my best to explain. Basically the tactic here is
	\begin{enumerate}
		\item Define the properties of the density matrix
		\item Define a basis of matrices
		\item Restrict the components of the vector space to yield valid (kinda) density matrices
		\item Create a uniform distribution of vectors within that space of valid matrices
	\end{enumerate}
	
	\section*{My Strategy}
	
	First, the properties of the density matrix (mathematically). Any matrix $\rho$ that satisfies
	\begin{align}
		\rho &= \rho^\dagger \label{eq:cond hermit}\\
		\Tr(\rho) &= 1 \label{eq:cond tr one}\\
		\rho &\geq 0 \label{eq:cond semidef pos}
	\end{align}
	is a valid density matrix. Note that \cref{eq:cond semidef pos} is just saying that the matrix is semidefinite positive (no negative eigenvalues).
	
	I was feeling inspired by the Bloch-vector parameterization approach mentioned in the paper, however if you actually try to implement the approach they talked about you'll find almost no Bloch-vectors yield a valid density matrix. I got to thinking about the basis $\{\mathbb{I}, \sigma_x, \sigma_y, \sigma_z\}$ for $2\times 2$ hermitian matrices, and realized we could construct a similar basis for $4\times 4$ hermitian matrices:
	\begin{align*}
		\{e_i\} = \{&\I\otimes\I, \I\otimes\sigma_x, \I\otimes\sigma_y, \I\otimes\sigma_z, \\
		&\sigma_x\otimes\I, \sigma_x\otimes\sigma_x, \sigma_x\otimes\sigma_y, \sigma_x\otimes\sigma_z \\
		&\sigma_y\otimes\I, \sigma_y\otimes\sigma_x, \sigma_y\otimes\sigma_y, \sigma_y\otimes\sigma_z \\
		&\sigma_z\otimes\I, \sigma_z\otimes\sigma_x, \sigma_z\otimes\sigma_y, \sigma_z\otimes\sigma_z\}
	\end{align*}
	Although looking really silly, this basis has some really stupidly wonderful properties for describing the density matrices. Plus, since it is a basis for Hermitian matrices \cref{eq:cond hermit} is automatically satisfied by any linear combination of $e_i$. Really quick, lets define $\bv{e} = \begin{pmatrix}e_0 & \cdots & e_{n^2-1}\end{pmatrix}^T$ so that any linear combination of the basis matrices is $\bv{v}\cdot\bv{e}$ for $\bv{v}\in\mathbb{R}^{n^2}$.
		
	Another wonderful property of this basis is that $\Tr(e_i)=0$ for $i\neq 0$, and $\Tr(e_0)=n^2$ (for $n$-qubit density matrices). This means that if we require $v_0=n^{-2}$ , then $\Tr(\bv{v}\cdot\bv{e})=1$, satisfying \cref{eq:cond tr one}. From here on out, I'm going to be \textit{very sloppy} about if I'm referring to the full $\bv{v}\in\mathbb{R}^{n^2}$ or dropping the first component and referring to $\bv{v}\in\mathbb{R}^{n^2-1}$, sorry about that!
	
	Furthermore, we can discuss the bounds of the components $v_i$ very nicely, since all $\sigma_i$ have eigenvalues of $\pm 1$, we must have $v_i\in[-1,1]$.
	
	To show the above fact robustly, we have to make note of perhaps the most important and useful fact of this basis: all $e_i$ are \textbf{orthogonal} under the Hilbert-Schmidt inner product! This is why we can restrict $v_i$ so cleanly, since
	\begin{align*}
		\langle e_j\rangle &= \Tr(e_j\rho) \\
		&= \Tr(e_j(\bv{e}\cdot\bv{v})) \\
		&= \Tr(e_j\sum_i e_iv_i) \\
		&= \sum_i v_i\Tr(e_je_i) \\
		&= \sum_i v_i\Tr(e_j^\dagger e_i) \\
		\langle e_j\rangle &= v_j
	\end{align*}
	Where the second to last line follows since all $e_i$ are Hermitian so $e_i = e_i^\dagger$. Since we know that the expectation value for any $\sigma_i$ is $\pm 1$, the expectation value for $\sigma_i\otimes\sigma_j$ (or equivalently, any $e_k$) is also $\pm 1$, which restricts the value of $v_i$ thanks to the math shown above.
	
	But here is the real kicker. Consider the Hilbert-Schmidt norm of one of these matrices.
	
	\begin{align*}
		||\bv{v}\cdot\bv{e}||_2 &= \sqrt{\Tr\{(\bv{v}\cdot\bv{e})^\dagger(\bv{v}\cdot\bv{e})\}} \\
		&= \sqrt{\Tr\left\{\sum_i\sum_j v_iv_je_i^\dagger e_j \right\}} \\
		&= \sqrt{\sum_i\sum_j v_iv_j \Tr\left\{e_i^\dagger e_j \right\}} \\
		&= \sqrt{\sum_i\sum_j v_iv_j \delta_{ij}} \\
		&= \sqrt{\sum_i v_i^2} \\
		&= ||\bv{v}||
	\end{align*}
	In fact, the orthonormal vector space described by the coefficients of these spin matrices perfectly encodes the Hilbert-Schmidt distances between matrices as the Euclidian norm of the difference of two vectors.
	
	This fact is super convenient, because now we can just generate an even volumetric distribution of vectors $\bv{v}\in\mathbb{R}^{15}$ and we will satisfy a even spacing according to the Hilbert-Schmidt norms, depending on how you want to define an ``even spacing."
	
	\section*{Conclusion and Outstanding Questions}
	Putting it all together: Now we know that we can generate these $\bv{v}$ that live inside the unit-cube in $\mathbb{R}^{15}$ (15 for the $n^2=16$ basis matrices, $-1$ since $v_0=n^{-2}$ is predetermined) and each correspond to a unique density matrix\footnote{I didn't show this fact, but it is well-known that a basis for $n$ qubit density matrices must have at least $n^2-1$ components. Since we have exactly that many components, there are no extra degrees of freedom.} and do a fantastic job of encoding Hilbert-Schmidt distances as Euclidian norms in $\mathbb{R}^{15}$.
	
	Perhaps the most important note at this point is that \cref{eq:cond semidef pos}, is not automatically satisfied by this generation approach, however it is \textit{far} more likely to be satisfied (as I learned from my code implementation) than it is to be satisfied with the Bloch-vector parameterization. However, that calls into question: why is it not always satisfied?
	
	Here are my outstanding questions:
	\begin{enumerate}
		\item Why is \cref{eq:cond semidef pos} not always satisfied? What is the subspace in $\mathbb{R}^{15}$ that is traced out by only valid density matrix vectors? What does it look like? What properties does it have? 
		\item For a one-qubit analogue to this approach, $||\bv{v}||\leq 1$ is the correct bound for $\bv{v}$. I'm almost certain the same thing holds here, and I can think of one way to prove it (using uncertainty principles) but it would help me sleep at night to know this is definitely the case.
		\item Just because an even volumetric distribution of $\langle\sigma_i\rangle$ for a single qubit implies a random sampling doesn't necessarily mean that the same holds for $n\geq 2$ qubits. I'm wondering if there is a way to be more rigorous about this?
		\end{enumerate}	
	
	
	
	
	
	
	
	
	
	
	
	
	
	

\end{document}