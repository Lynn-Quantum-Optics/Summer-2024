\section{Jones Matrices}
	To describe the linear manipulation of photons by optical components, we use Jones matrices. If an optical component has a Jones matrix $\hat{J}$ then the quantum state vector $\ket{\psi}$ will become $\ket{\psi'}=\hat{J}\ket{\psi}$ after passing through it.
	
	\subsection{Half Wave Plate}
	A half wave plate with it's fast axis horizontal has the Jones matrix
	\begin{equation}
		\HWP_0=\begin{pmatrix}
			1 & 0 \\ 0 & -1
		\end{pmatrix}
	\end{equation}
	To emulate rotating the plate you can apply the inverse rotation matrix to the light, then rotate the result by the ordinary rotation matrix. Note these rotation matrices are given by
	\begin{align}
		R(\theta) = \begin{pmatrix}
			\cos\theta & -\sin\theta \\ \sin\theta & \cos\theta
		\end{pmatrix} \\
		R^\dagger(\theta) = \begin{pmatrix}
			\cos\theta & \sin\theta \\ -\sin\theta & \cos\theta
		\end{pmatrix}
	\end{align}
	And so the Jones matrix for a half wave plate with it's fast axis an angle $\theta$ from the horizontal is given by
	\begin{equation}
		\HWP(\theta) = R\;\HWP_0\;R^\dagger = \begin{pmatrix}
			\cos 2\theta & \sin 2\theta \\ \sin 2\theta & -\cos 2\theta
		\end{pmatrix}
	\end{equation}
	
	\subsection{Quarter Wave Plate}
	A quarter wave plate applies a phase shift of $-i$ to the component along it's slow axis, and so if the fast axis is aligned with the horizontal then the Jones matrix is
	\begin{equation}
		\QWP_0 = \begin{pmatrix}
			1 & 0 \\ 0 & i
		\end{pmatrix}
	\end{equation}
	And applying the same rotation trick, if its fast axis lies an angle $\theta$ from the horizontal then
	\begin{equation}
		\QWP(\theta) = R\;\QWP_0\;R^\dagger = \begin{pmatrix}
			\cos^2\theta + i\sin^2\theta & (1-i)\sin\theta\cos\theta \\
			(1-i)\sin\theta\cos\theta & \sin^2\theta + i\cos^2\theta
		\end{pmatrix}
	\end{equation}
	
	\subsection{Quartz Plate}
	
	The quartz plate in our setup works by changing the distance that the light travels through it. By rotating the plate to make an angle $\theta$ with the horizontal (perpendicular to the optical axis), then light will travel a distance $d=1+d_0\sec\theta$ and the vertical component will incur a relative phase difference of $(k_2-k_1)(1+d_0\sec\theta)$. In other words, we can model the quartz plate with the jones matrix
	\begin{equation}
		\QP = \begin{pmatrix}
			1 & 0 \\ 0 & e^{i\phi}
		\end{pmatrix}
	\end{equation}
	Where
	\begin{equation}
		\phi = A + B\sec\theta
	\end{equation}
	for some $A,B\in\mathbb{R}$.